\chapter{Unparser}
The unparser is responsible to find all valid parse trees for the current language model and select the best. The process of creating an parse tree from an AST is called ``unparsing'' in this thesis. The creation of a token or character stream from the parse tree is called pretty printing. The actual unparsing process is twofold:
\begin{itemize}
	\item find all combinations of production rule applications that uniquely distinguish valid words
	\item create a parse tree according to the most promising combination. 
\end{itemize}
 
The suggested solution is close the one implemented in Xtext. Xtexts solution, which is described in \ref{xtxt:ptc}, stops after finding one combination. 
To enable multiple representations, the alternatives need to be made available. This is done by the notation model, which is described in \ref{chp:NotMM}. The notation model is able to be equivalent to a parse tree, but also allows to refer to following productions, called hints. EMFs \code{ECrossReferenceAdapter} virtually creates bidirectional references from unidirectional. This allows the use or reuse of notation model elements which refer to AST nodes just by getting all referring \code{EObjects} and filtering a type.

\todo{blueprints based on s-attributes. Tree struct based on SAttrs.}

\subsection {Hints}
A hint is a reference to a following production. So with hints, a node in the parse tree is not only labeled by its non terminal, but also by its production. Hints specify a single production without referring to a parse tree part, which is important, if multiple productions are possible. Thus, hints do not refer to actual content but define the structure of the content. Regarding the unparser, the AST is present, but it is ambiguous which structure to use to hold its contained data.  

\section{Unparse Steps}
The basic steps of the unparser are:
\begin{enumerate}
	\item compute all possible solutions. This results in a forest of parse tree blueprints.
	\item determine the best tree. Which criteria compose the best tree are explained below.
	\item produce a parse tree for the best tree. 
	\item compress the forest. For each branch branching the best tree, set the first node on the branch as an alternative representation of the best trees node it branches from.
\end{enumerate}

Criteria that determine the result tree are ranked from most to least important:
\begin{enumerate}
	\item use notation model hints, if exits. Hints are likely set by the user and thus have top priority.
	\item number of similar productions to the previous unparse or parse, which is saved in the notation model. xxx Keeps stable.
	\item user preferences
	\item language designer preferences.
	\item number of default values used as negative criteria.
	\item number of direct EObjects required as negative criteria.
\end{enumerate}

\todo{blueprints 2 parsetree}

The suggested behavior is that the user does not specifies the whole alternative representation, but selects an alternative, let the unparse create and display the new representation and iteratively refines the presentation. 