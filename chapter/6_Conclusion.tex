%%%%%%%%%%%%%%%%%%%%%%%% Conclusion %%%%%%%%%%%%%%%%%%%%%%%%
\chapter{Conclusion}
% diff txt rep
\paragraph{Alternative Textual Representations}
By conceptually extending the existing Parse Tree Constructor of Xtext, different textual representations are possible. The possible ambiguous mapping from model to Parse Tree is leveraged to allow different, context free grammar conform representations. For a user, a high degree of ``valid'' ambiguity is desirable, because it provides him with a higher variety of semantically equivalent representations. The extension of the Parse Tree Constructor requires it to calculate all instead of one possible valid solutions for a part of the word. Considering that the current Parse Tree Constructor uses Backtracking, a high degree of ambiguity is unfavorable for Parse Tree Constructor's runtime. Plain backtracking will not scale in calculating all solutions, especially regarding the combinatorial explosion of alternative combined list elements, so for practical use an improved algorithm is desirable. The alternative textual representations are alternative Parse Trees, thus the representations are not limited to different token values, but allow a different \emph{structural}. This can be leveraged at grammar design time to avoid manual model transformations. 

\paragraph{Structured Characters}
The use of the invented EMF Lexers Structured Characters to enable \code{EObject}s on the character stream \emph{as Terminals} releases the grammar from the burden to \emph{describe} all model elements, but still requires to \emph{declare} their occurrence. This hinders the user to textually change or create those \code{EObject} Terminals, so if the model should be ever be editable outside a specially adapter editor, their use should be avoided. Unhandled Structured Characters cause severe usability issues, because from the users perspective, they present themselves as $\square$, a squared ``black'' box.

\paragraph{Sentential Character based Graphical Editors}
Sentential forms of the Parse Tree can be persisted and used instead of the original input, \cite{iSW}. Post processing of the grammar together with Structured Characters allows the use of sentential forms on the character stream, but results in larger Parse Tables. Because a sentential node, which is the root of the persisted subtree, contains information which production part in the grammar it represents, the Parse Tree Constructor could produce a valid sentential form in conjunction with the corresponding language model elements at any time, which is rendering actual parse subtree construction for sentential characters unnecessary. The Sentential Character is valid, if the corresponding language model elements conform to the constraints of its production part. While presenting the character stream, a graphical editor can substitute the Sentential Character and operate on the model elements it holds, as long as commits keep the model elements in a state which conforms to the production part constraints. This procedure allows to integrate graphical editors into texts which still \emph{conform to a context free grammar}. During grammar design time, it is not required to regard graphical editors. Graphical editors can substitute \emph{any} symbol, as well as \code{Sequence}s, which enables fine grained editors with potential overlapping functionality. To determine a specific graphical editor for a sentential character, the referred \code{EObject} is specialized by the Parse Tree Constructor. In order for this specialization to work, the graphical editor has to register itself at the Parse Tree Constructor with its specialization type, the grammar part, for example for which \code{Symbol} it provides an alternative representation to and optionally which further constraints the corresponding language model elements must match. Thus, incremental visual domain specific language development from an internal textual domain specific language is possible. The graphical editor extensions of the language are probably locked to the editor framework. Because language extensions in general are also locked to the language metamodel, a common metamodel for a textual language should be shared across the editor framework.


\paragraph{Effect of Unstable Models}
The characteristic that language model elements are replaced instead of updated for textual changes is called \emph{unstable}, because of its unstable effect for references or URI based references, so called proxies. Because every parsing technique used by the identified EMF related textual editing tools, EMFText, Xtext and Textual Editing Framework results in unstable models, the effect of unstable models to the presented solutions is of interest for practical purposes. 
\begin{itemize}
	\item First, the language model decorating, thus referencing Notation model has to be rebuild on every textual update. Thus, every information which can not be derived from the textual representation is lost.
	\item  Second, references of the \code{EObject}s in the Structured Characters that refer to \code{EObject}s that might get updated due to textual changes are unstable. This can be circumvented by erasing the reference and use a textual identification stored in an \code{Annotation} to resolve the reference like any inner textual reference: programmatically. Changes to the identification of the defining occurrence, the referred target, must be taken over to the referring occurrence, the textual identification in the \code{Annotation}, by the editor.
	\item Third, Sentential Characters reference to the language model must be handled, where Parse Tree construction was omitted due to the existence of the Parse Tree Constructor and the referred \code{EObject}. To circumvent the merging of an old and new version of the language model by the use of the Notation model, a containment reference instead of a cross reference from the \code{EObjectProduction} to the referred language \code{EObject} is suggested. Contained Notation model elements depend on data of the first directly or indirectly containing \code{EObjectProduction}, if they are not one itself. Thus, the use of Sentential Characters is restricted to \code{EObjectProduction}, or to parts of the attributed grammar which produce or refer to an \code{EObject} itself. 
\end{itemize}

\paragraph{EMF based Textual Frameworks}
At the current state of existing EMF based textual frameworks, only unstable models are produced. Thus manually set cross-references, for example by a graphical editor, must be \emph{manually maintained each time the textual representation is changed}. This work is repetitive, cumbersome, error prone and not maintainable for large, frequently changed models. This basically makes manual set cross references, so all cross references set by an graphical editor, unusable in the general case, where no intrinsic identification of language elements exist in the language. The integration of existing textual frameworks in EMF is \emph{heavily limited in regard to cross references}. Cross references are one of the key advantages of languages defined by Metamodels over context free grammars. This problem is introduced by the use of certain existing Parser generators, ANTLR for Xtext and EMFText and RunCC \footnote{\raggedright \url{http://runcc.sourceforge.net/}} for TEF, by the EMF based textual frameworks. These Parser generators are designed support the creation of a language based on a context free grammar. With the absence of cross references, two equal Parse Trees could be handled as interchangeable or identical. The created references in practice, like symbol tables, are automatically derived by from the textual representation which solely describes the abstract syntax, so this problem does not raise for the purpose the parser generators were designed for. On the other hand parser technologies were developed, which optimally reuse existing parse trees, for the purpose to skip potentially CPU intensive post processing. The Parse Tree reuse of incremental parsers would allow manually set cross-references. The choice of existing EMF based textual frameworks to use non incremental Parsers for EMF integration instead of incremental Parsers like \cite{xglr} is questionable.


Despite the problem of unstable models, alternative graphical representations of grammar conform word parts with Sentential Characters remains feasible, making graphical internal domain specific languages possible. 
$\square$
