\chapter{MyAddedValue}
\section{Text Editor}
as reasoned in \todo{REF XTEXT Conclusion} it is just possible to view the unaugmented textual serialized representations of language without potential loss of referential integrity. As a result the modeled data must be described entirely by the word. To seamlessly integrate textual languages in EMF, referential integrity must be kept. To maintain referential integrity, extrensic IDs could be added  to the token stream. The text editor must be able to handle IDs or the user editing the serialized representation must comply to a contract. A simple example is that an ID is kept unique. Allowing exactly this case might be beneficial because the double occurance of an ID indicates the semantics of a change: that one element was copied. A special integration layer between the token stream leaving the text editor and the token stream entering the model parser could handle this case. This is not always true, for example if the original element was refered and the clone can not be distingushed from the original. \\
The textual model editor must be able to handle IDs in general, not just the ones attached to lexemes. 


\section{ID Token}
\subsection{General Requirements}
The job of an ID token is to uniquely identify something in a serializeable context. 
The ideal characteristics are:
\begin{itemize}
	\item bindable and existence depedant on another token or standalone
	\item unlimited number of identifiers
	\item atomic, ID integritiy should be perserved, unbreakable
	\item standardised, long term guarantee
	\item ubiqious available
	\item abstract characters: fast distinguisable from non ID tokens
	\item private / missinterpretable / non amigious in its use
\end{itemize}

\subsection{Unicode Private-Use Characters}
A possible solution offers Unicode \cite{Unicode}. Unicode is a universal character encoding standard for consistent encoding and exchange of text data. It is the default encoding of HTML and XML and is implemented in all modern operating systems. It specifies a numeric value (code point) and name for each character. Unicode was developed in conjunction with the Universal Character Set and can be represented by widely available encodings like UTF-8. The Unicode Standard defines private-use characters, which interpretation is not specified and is determined by a private agreement among cooperative users. For example Apple uses a private character to present it's apple logo. A application specific changed interpretation of for example the character representing the apple logo is valid, as it specifies its intended behaviour according to a private agreement.\\
The private use characters are intended for softwaredevelopers. They can be compared to the ideal characteristics:
\begin{itemize}
	\item they are stand alone characters and are not bindable per se.
	\item The number of identifiers is limited to 137,468, in which 6,400 are in the private use area U+E000 to U+F8FF and 65,534 are in each supplementary private use Area-A and Area-B. 
	\item They are single characters and thus atomic. 
	\item Unicode is standardised and private-use characters are permanently designated for private use.
	\item Unicode, especially the UTF-8 enconding is wide spread and nearly ubiqious avaiable on modern personal computer platforms.
	\item There are three code-blocks for private-use characters. Three range check for a code point is in sufficient to determine its private-use.
	\item Private-use characters are, as the name states, explicitly designed for private use.
\end{itemize}

The restriction of a limited number of available identifiers could be solved by implementing a non-standard character encoding, probably a variable-with enconding. 