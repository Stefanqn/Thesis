\chapter{non}


\section{Proceeding} 
This section describes the proceedings of how the results were achieved.

The following observations during research were made:
\begin{itemize}
	\item Every non-metamodel based tool uses its own, specific approach for abstract language data structure transformation, validation, persistence and editing.
	\item In the Graphical Modeling Framework (GMF), the \emph{Notation model} generically encapsulates the state of the visual representation. The \emph{Extensible Type Registry} determines a type or specialization type which is used to guide Notation model element creation. 
	\item After a short evaluation of the Meta Programming System (MPS), the restricted usability of \emph{projectional textual editing is undesirable} for the target of this thesis. The text editing concept of MPS is template or form based, not free textual.
	\item The report \cite{Barista} about Barista shows the successful use of a variant of Harmonia's incremental parsing algorithms in projectional editors to achieve \emph{stable abstract data structures}.
	\item Harmonia's incremental parsing algorithm for GLR parsing requires the Parser to handle \emph{sentential forms}.
	\item Proxima uses a \emph{Structured Token for graphical presentation}, but does not allow it to be directly edited.  Its seven layer architecture with bidirectional model transformations between each allows high flexibility. Dependant layers were considered undesirable, especially regarding possible user interaction on different layers, with the complexity to maintain bidirectionality by two unidirectional transformations and updating instead of replacing model elements. Several model transformation languages for EMF exist, so structural transformations are possible and separated if required. 
\end{itemize}

The Xtext framework was examined afterwards in respect to adding alternative representations and related problems. The following central perceptions were made:
\begin{itemize}
	\item Structure of textual representation is coupled to the structure of the model. This characteristic is condoned with regard to the availability of EMF model to model transformations.
	\item The used, non incremental parsing technology results in unstable models. This characteristic is not considered until its impact on the solution is in the \ref{cha:discussion} discussion.
	\item The grammar must completely describe the model. 
	\item There is no isomorphism from grammar production to \code{EObject}, which means multiple representations of the same \code{EObject} are possible.
	\item The Parse Tree Constructor determines the production used for model to text transformations. Different textual representations are already possible, but are not accessible by the user or language designer. \\
\end{itemize}
 
 
The first conceptional improvement of this thesis allows \emph{switchable textual representations}. This has been solved by extending the parse tree constructor and by providing a hint to guide the parse tree construction. A hint identifies the next production part.
The parse tree constructor is conceptually extended to
\begin{enumerate}
	\item determine all valid solutions for a node and assign them to it and
	\item to prefer certain hints while the resulting parse tree is constructed.
\end{enumerate}
For combined use with the parse tree constructor a Notation model was developed to
\begin{itemize}
	\item provide a proper place to assign alternative presentations and
	\item to uniquely identify productions or parts of it as hints.
\end{itemize}
To uniquely distinguish a specific tree construction for hints, the structure resembled a parse tree structure. The notation model was extended to substitute the parse tree accordingly.\\


\emph{Structured Tokens} highlighted the usage of nested structured data, like \code{EObject}s as Tokens, which triggered the second conceptional improvement. To integrate Structured Tokens in a textual serialization, a Structured Character concept was developed. It is based on the basic idea to use a private, unique character as a key for a map entry of a data structured persisted in an adjacent resource, like a file. The designated solutions are UTF8 private use characters and a wrapping Lexer which resolves and types the \code{EObject}s from the adjacent resource. This enables the use of \code{EObject}s as atomic characters on the character stream. \\

The possibility to use \code{EObject}s on the character stream enables the use of Notation model elements. Notation model elements represent parts of the Parse Tree or \emph{Sentential Forms}. To allow their use as \emph{Sentential Token}s, two improvements are necessary:
\begin{itemize}
	\item The extended Lexer must assign the Token name of the Sentential Token properly. The name can not be determined by the generic Notation model element's type, but by general Notation Metamodel specific rules, which resolve the grammar relation of the Notation element.
	\item The input grammar for the used Parser generator must be post processed, so that each non-terminal on the right hand side of a rule is replaced by a new alternative. The new alternative consists of the original non-terminal $\overline{or}$\footnote{\raggedright Metalinguistic or. See \cite{BNF}} a terminal with the synthetic name of the grammar related element, which would be assigned by the Lexer.
\end{itemize}
Serialized Sentential Tokens are named \emph{Sentential Characters}. Sentential Characters alone simply hide the data structure of a production part in one character and perfectly integrate in the language described by the context free grammar. \\

To improve the usability of \emph{Structured Characters} for Presentation, the following enhancements are suggested:
\begin{itemize}
	\item An Extensible Parse Tree Constructor, which optionally specializes the determined type based on additional constraints. This is similar to GMFs Extensible Type Registry.
	\item To omit Parse Tree construction if a specialized presentation type was found. A graphical editor could then safely operate on the Language model elements.
\end{itemize}
A user interface could create a graphical editor at the place of his specific specialized presentation character and operate on the Language model elements it hides.\\


Finally, the problem of unstable models and its impact on the suggested solution is resumed as part of the discussion.
