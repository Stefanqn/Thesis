\begin{abstract}
%Leveraging the increasing number of physical and virtual cores is one of the big challenges in Software development. Dealing with inversion of control imposed by reactive architectures to achieve scalability is further increasing the complexity of software. Out of the multitude of concepts and patterns, this thesis focuses on the DataFlow Model to create concurrent programs, whose control-flow is determined by the data itself, allowing to write programs with an uncluttered logical flow. Based on the hybrid object oriented and functional programming language Scala and its advanced feature of transformation into Continuation-Passing-Style, this work presents the design and implementation of a framework that allows to embed concurrent DataFlow environments into Scala projects.
One of the biggest challenges in software development is leveraging the increasing number of physical and virtual cores. Reactive architectures promise improved scalability, but impose an inversion of control which further increases the complexity of parallel software. Out of the multitude of concepts and patterns, this thesis focuses on the DataFlow Model as a basis for concurrent programming. In this model control flow is determined primarily by the data itself, allowing the creation of programs with an uncluttered logical flow. This thesis builds on the hybrid object oriented and functional programming language Scala, and in particular its advanced feature of automatic transformation into Continuation-Passing-Style. It describes the design and implementation of the ScalaFlow framework, which allows concurrent DataFlow environments to be embedded into Scala projects.
\end{abstract}