%%%%%%%%%%%%%%%%%%%%%%%%%%%%%%%%%%%%%%%%%%%%%%%%%%%%%%%%%%%%%%%%%%%%%%%
\documentclass
  [ twoside            % beidseitiger Druck
  , BCOR=10mm          % Bindekorrektur
  , openright          % Kapitel beginnen auf einer rechten Seite
  , listof=totoc       % Verzeichnisse im Inhaltsverzeichnis
  , bibliography=totoc % Literaturverzeichnis im Inhaltsverzeichnis
  , parskip=half       % Absätze durch einen vergrößerten Zeilenabstand getrennt
%  , draft              % Entwurfsversion
  ]{scrreprt}          % Dokumentenklasse: KOMA-Script Buch

\usepackage{scrhack}

%%%%%%%%%%%%%%%%%%%%%%%%%%%%%%%%%%%%%%%%%%%%%%%%%%%%%%%%%%%%%%%%%%%%%%%
% Packages
%%%%%%%%%%%%%%%%%%%%%%%%%%%%%%%%%%%%%%%%%%%%%%%%%%%%%%%%%%%%%%%%%%%%%%% 
\usepackage{ifpdf}
\ifpdf
  \usepackage{ae}               % Fonts für pdfLaTeX, falls keine cm-super-Fonts installiert
  \usepackage{microtype}        % optischer Randausgleich, falls pdflatex verwandt
  \usepackage[pdftex]{graphicx} % Grafiken in pdfLaTeX
\else
  \usepackage[dvips]{graphicx}  % Grafiken und normales LaTeX
\fi

\usepackage[utf8]{inputenc}         % Input encoding (allow direct use of special characters like "ä")
\usepackage[english]{babel}
\usepackage[T1]{fontenc}
\usepackage[automark]{scrpage2} 	% Schickerer Satzspiegel mit KOMA-Script
\usepackage{setspace}           	% Allow the modification of the space between lines
\usepackage{booktabs}           	% Netteres Tabellenlayout
\usepackage{multicol}               % Mehrspaltige Bereiche
\usepackage{quotchap}               % Beautiful chapter decoration
\usepackage[printonlyused]{acronym} % list of acronyms and abbreviations
\usepackage{subfig}                 % allow sub figures
\usepackage{tabularx}              % Tabellen mit fester Breite

% Layout
\pagestyle{scrheadings}
%\pagestyle{empty}
\clubpenalty = 10000
\widowpenalty = 10000
\displaywidowpenalty = 10000

\makeatletter
\renewcommand{\fps@figure}{htbp}
\makeatother

%% Document properties %%%%%%%%%%%%%%%%%%%%%%%%%%%%%%%%%%%%%%%%%%%%%%%%
\newcommand{\projname}{TextPart}
\newcommand{\titel}{TextPart}
\newcommand{\untertitel}{Approach to mix EMF based visual and text editors}
\newcommand{\Datum}{September 13th, 2012}

\ifpdf
  \usepackage{hyperref}
  \definecolor{darkblue}{rgb}{0,0,.5}%{0,0,.5}
  \hypersetup
  	{ colorlinks=true
  	, citecolor=darkblue
  	, breaklinks=true
    , linkcolor=darkblue
    , menucolor=darkblue
    , urlcolor=darkblue
    , pdftitle={\projname -- \untertitel}
    , pdfsubject={Master's Thesis}
    , pdfauthor={Stefan Kuhn}
    }
\else
\fi

%% Listings %%%%%%%%%%%%%%%%%%%%%%%%%%%%%%%%%%%%%%%%%%%%%%%%%%%%%%%%%


\usepackage{listings}
\KOMAoptions{listof=totoc} % necessary because of scrhack
\renewcommand{\lstlistlistingname}{List of Listings}
\lstset
  { basicstyle=\small\ttfamily
  , breaklines=true
  , captionpos=b
  , showstringspaces=false
  , keywordstyle={}
  }

\lstnewenvironment{inlinehaskell}
{\spacing{1}\lstset{language=haskell,nolol,aboveskip=\bigskipamount}}
{\endspacing}

\lstnewenvironment{inlinexml}
{\spacing{1}\lstset{language=XML,nolol,aboveskip=\bigskipamount}}
{\endspacing}

\newcommand{\haskellinput}[2][]{
  \begin{spacing}{1}
  \lstinputlisting[language=Haskell,nolol,aboveskip=\bigskipamount,#1]{#2}
  \end{spacing}
}


\newcommand{\haskellcode}[2][]{\mylisting[#1,language=Haskell]{#2}}

\newcommand{\mylisting}[2][]{
\begin{spacing}{1}
\lstinputlisting[frame=lines,aboveskip=2\bigskipamount,#1]{#2}
\end{spacing}
}

%% additional commands
\newcommand{\todo}[1]{\marginpar{\textbf{TODO}} \textcolor{red}{#1}}
\newcommand{\code}[1]{\texttt{#1}}

%%% Kai >>

\lstdefinelanguage{scala}{
  morekeywords={abstract,case,catch,class,def,%
    do,else,extends,false,final,finally,%
    for,if,implicit,import,match,mixin,%
    new,null,object,override,package,%
    private,protected,requires,return,sealed,%
    super,this,throw,trait,true,try,%
    type,val,var,while,with,yield},
  otherkeywords={=>,<-,<\%,<:,>:,\#,@},
  sensitive=true,
  morecomment=[l]{//},
  morecomment=[n]{/*}{*/},
  morestring=[b]",
  morestring=[b]',
  morestring=[b]"""
}

\definecolor{lstbg}{gray}{.95}

\lstnewenvironment{scala}[1][]
  {\spacing{1}\lstset{language=scala,frame=single,backgroundcolor=\color{lstbg},aboveskip=\bigskipamount,numberbychapter=true,#1}}
  {\endspacing}


%% ich <<< ==============================================
\usepackage{color}
\usepackage{epsfig}
\usepackage{amsmath}
\usepackage{epstopdf}
\usepackage{import}
\usepackage{placeins}

\lstdefinelanguage{Xtext2}{
  morekeywords={abstract,case,catch,class,def,%
    do,else,extends,false,final,finally,%
    for,if,implicit,import,match,mixin,%
    new,null,object,override,package,%
    private,protected,requires,return,sealed,%
    super,this,throw,trait,true,try,%
    type,val,var,while,with,yield},
  otherkeywords={=>,<-,<\%,<:,>:,\#,@},
  sensitive=true,
  morecomment=[l]{//},
  morecomment=[n]{/*}{*/},
  morestring=[b]",
  morestring=[b]',
  morestring=[b]"""
}


\lstdefinelanguage{xtext}{
  keywords={import, grammar, as, terminal, returns, generate, with}, 
  morecomment=[l]{//},
  morecomment=[s]{/*}{*/},
  morestring=[s]{'}{'},
  morestring=[s]{"}{"},
  alsoletter={&,\&},
  alsoother={&,\&}
}

\lstnewenvironment{xtxt}[1][]
  {\spacing{1}\lstset{language=xtext,frame=single,backgroundcolor=\color{lstbg},aboveskip=\bigskipamount,numberbychapter=true,#1}}
  {\endspacing}
  
% \newcommand{\kode}[1] {\begin{lstlisting}^^M foo \end{lstlisting}}
%{\lstset{language=xtext,{#1}}}

\newcommand\kodeN[1]{\lstinline[basicstyle=\tt]{#1}} 
\def\kode{\lstinline[basicstyle=\ttfamily]} 

% \newcommand{\kode2}[1]{\begin{xtxt}#1\stop{xtxt}}
%{\start{xtxt}#1}}